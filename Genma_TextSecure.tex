\documentclass{beamer}
\mode<presentation> {
%\usetheme{Madrid}
%\usetheme{default}
\usepackage{color}
\definecolor{bottomcolour}{rgb}{0.21,0.11,0.21}
\definecolor{middlecolour}{rgb}{0.21,0.11,0.21}
\setbeamercolor{structure}{fg=white}
\setbeamertemplate{frametitle}[default]%[center]
\setbeamercolor{normal text}{bg=black, fg=white}
\setbeamertemplate{background canvas}[vertical shading]
[bottom=bottomcolour, middle=middlecolour, top=black]
\setbeamertemplate{items}[circle]
\setbeamertemplate{navigation symbols}{} %no nav symbols
\setbeamercolor{block title}{use=structure,fg=white,bg=structure.fg!50!red!50!blue!100!green}
\setbeamercolor{block body}{parent=normal text,use=block title,bg=block title.bg!5!white!10!bg,fg=white}
\setbeamertemplate{navigation symbols}{}
}

\usepackage{graphicx} 
\usepackage{booktabs} 
\usepackage[utf8]{inputenc}  
\usepackage[T1]{fontenc}  
\usepackage{geometry}     
\usepackage[francais]{babel} 
\usepackage{eurosym}
\usepackage{verbatim}
\usepackage{ragged2e}
\justifying

\input{cc_beamer}

\title[TextSecure]{TextSecure} 
\author{Genma}

\begin{document}

%% Titlepage
\begin{frame}
	\titlepage
	\vfill
	\begin{center}
		\CcGroupByNcSa{0.83}{0.95ex}\\[2.5ex]
		{\tiny\CcNote{\CcLongnameByNcSa}}
		\vspace*{-2.5ex}
	\end{center}
\end{frame}

%------------------------------------------------
\begin{frame}
\frametitle{À quoi sert TextSecure ? }

\justifying{
C'est un logiciel de messagerie instantanée, conçu pour être très simple d'usage, pour servir de remplaçant « tel quel » aux logiciels SMS actuels, tout en offrant une meilleure protection de la vie privée, dans certains cas. 
\begin{itemize}
\item TextSecure remplace l'application SMS par défaut.
\item TextSecure offre un bon compromis entre facilité d'usage et sécurité.
\end{itemize}
}
\end{frame}

%------------------------------------------------
\begin{frame}
\frametitle{TextSecure du point de vue utilisateur}

 \justifying{
L'utilisateur s'en sert comme de l'outil SMS par défaut. \\~\\
Le principal changement est que certains messages vont être marqués d'un cadenas : 
\begin{itemize}
\item 
si le correspondant utilise également TextSecure, les messages sont chiffrés automatiquement avec sa clé.
\end{itemize}
Si on est simple utilisateur, on a donc un peu plus de vie privée, et on en aura de plus en plus au fur et à mesure que TextSecure se répand.
}
\end{frame}


%------------------------------------------------
\begin{frame}
\frametitle{Comment fonctionne TextSecure?}

 \justifying{
TextSecure peut utiliser deux transports différents le SMS et la liaison « données » du smartphone (ce second transport se nomme PUSH dans la terminologie TextSecure, et est utilisé par défaut pour les correspondants qui ont également TextSecure). 
\\~\\
Le mode PUSH peut être intéressant si on a un quota SMS limité mais pas en Internet, et le mode SMS dans le cas contraire. 
\\~\\
Les messages apparaissent en vert traditionnel quand ils ont été transportés en SMS et en bleu autrement.
}

\end{frame}

%------------------------------------------------
\begin{frame}
\frametitle{Chiffrement de tous les SMS}

 \justifying{
TextSecure peut aussi chiffrer toute la base des SMS reçus et stockés. Ainsi, même si on vole votre téléphone, vos communications resteront sûres. 
\\~\\Attention : si vous oubliez la phrase de passe, tout est fichu. Pensez à sauvegarder.
}

\end{frame}

%------------------------------------------------
\begin{frame}
\frametitle{Les limites de TextSecure}

 \justifying{
Lles métadonnées de la communication sont en clair. Des tiers peuvent donc savoir qui écrit à qui et quand, même si le contenu des messages est chiffré. 
\\~\\
Les metadata sont connues d'OpenWhispe, si on utilise PUSH 
\\~\\
ou de l'opérateur mobile, si on utilise le SMS traditionnel.
}

\end{frame}


%------------------------------------------------
\begin{frame}
\frametitle{Peut-on faire confiance à TextSecure? }

 \justifying{
TextSecure est un logiciel libre : le code est disponible, et des experts en sécurité l'ont déjà lu et personne n'a encore trouvé de défaut significatif. 
\\~\\
Le code du serveur utilisé par OpenWhisper est également disponible en ligne.
\\~\\
Mais cela s'avère moins utile car on ne peut pas vérifier que le serveur effectif exécute bien ce code.
}

\end{frame}


%------------------------------------------------
\begin{frame}
\frametitle{Comment valider sa correspondance?}

 \justifying{
À la première communication, TextSecure fait confiance, c'est un système, dit TOFU (Trust On First Use), qui a l'avantage d'être trivial d'utilisation. 
\\~\\
Si le correspondant réinitialise sa clé mais, d'après la documentation, TextSecure vous prévient.
\\~\\
Si on veut, on peut toujours vérifier, lors d'une rencontre AFK, la clé de son correspondant, soit en la lisant à l'écran, soit via des codes QR. 
}

\end{frame}


%------------------------------------------------
\begin{frame}
\frametitle{Les limites de TextSecure}

 \justifying{
Aucun moyen d'enregistrer le fait qu'on a vérifié, afin, par exemple, d'afficher les messages ultérieurs d'une couleur différente si le correspondant a ainsi été solidement authentifié. 
\\~\\
On ne peut pas envoyer un message à plusieurs personnes simplement (il faut d'abord créer un groupe statique).
}

\end{frame}

%------------------------------------------------
\begin{frame}
\frametitle{En savoir plus?}
 \justifying{
Pour aller plus loin, la société qui développe TextSecure, OpenWhisper, a une bonne FAQ.
}
\end{frame}

\end{document}
